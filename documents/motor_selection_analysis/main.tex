\documentclass[11pt,table]{article}
\usepackage{preamble}

\title{Motor Selection Analysis}
\date{March 2020}

\begin{document}
\maketitle

\textbf{Disclaimer}: A lot of this is probably model-rocketry-101, and will not be 100\% accurate. I've never launched a
model rocket before, so if anything sounds completely off, please let me know.

\section*{Overview}
The two main factors that we don't have control over are:

\begin{itemize}
	\item The external environment (the structure of the atmosphere, weather conditions on launch day).
	\item The internals of the motor we choose to use (construction, behaviour), because of the category we are
		competing in (COTS).
\end{itemize}

We have design freedom over the remainder of the launch vehicle (LV). This is not complete because of the constraints
set out by ESRA, but these are relatively high level. Once we've finalised a motor, the design space for the rest of the
LV is opened up. It seems possible to derive almost all remaining design questions from the motor and the competition
goal. Without providing links to all design decisions that will have to be made, the following lists a few examples.
These allow cascading information to be determined. Further into design, sub-problems can be properly defined.

\begin{itemize}
	\item Diameter gives us upper and lower bounds on the LV's total length (using stability, caliber, fineness ratio
		recommendations/constraints for target-apogee flight). This can in turn be used to make general statements about
		and define constraints for the air frame and then space we have to store avionics hardware, recovery hardware,
		payload etc...

	\item Mass and thrust data gives us a hard upper bound on the mass of the remainder of the LV if we're to reach
		10'000ft, which all feeds into the paragraph above.

	\item Price gives us a rough estimate of remaining budget to work with (even if this is subject to change).    

	\item $\cdots$
\end{itemize}

\section*{Motor Parameters}
The following section outlines some motor parameters that should be used to inform our choice. Some of these are given
on data sheets online, others need to be calculated.

\subsection*{Available with data sheets}
\begin{itemize}
	\item Type (single-use, reloadable, hybrid)
	\item Delay time(s)/whether it even has a delay charge or not (booster)
    \item Average thrust
    \item Maximum thrust
    \item Thrust-vs-time curve
    \item Total impulse and impulse class
    \item Burn time
    \item Diameter
    \item Length
    \item Total mass
    \item Propellant mass (and how this changes over the burn)
    \item Centre of gravity (and how this changes over the burn)
\end{itemize}

These parameters can easily be obtained from a \href{http://wiki.openrocket.info/RSE_File}{.rse file}.

\subsection*{Determined independently}
\begin{itemize}
	\item Allowable - does the SAC allow the use of this motor? Any motors with a total impulse greater than 40'960 Ns
		cannot be used. This will rule out higher-power motors.
	\item Sufficiency - in the best case, is it physically possible for this motor to reach 10'000ft with a 4kg payload?
		This will rule out lower-power motors.
	\item Overall consistency between units (focusing on but not limited to the thrust curve) - what uncertainty is
		there around the thrust curve provided in the data sheet? Are there multiple sets of test data for this motor
		that support each other? What is the source of the test data? Has the manufacturer supplied a measure of
		standard deviation for any parameters? How many tests have been carried out to determine values for the
		datasheet?
	\item Reliability - are there any instances where this motor has been known to CATO?
	\item Price - is this motor too expensive?
\end{itemize}

\section*{Selection Criteria}
Given the complete set of parameters above and the competition goal, in what order should we consider the parameters when
selecting the optimal motor? This list is ordered based on priority and describes the ideal value of each parameter, so
the process of selecting a motor becomes an optimisation problem.

\begin{enumerate}
    \item \underline{Allowable} \\
	Must be allowed
    
    \item \underline{Sufficiency} \\
	Must be able to reach $\geq$ 10'000ft carrying $\geq$ 4kg of extra mass (at this stage, the required payload is the
	only mass information we have)
    
    \item \underline{Reliability} \\
    Maximise \\
	It's not obvious how this could be quantified. It could involve the number of reported failures online in blogs,
	videos, forums, papers. Is it correct to assume that modern motors are more reliable than older motors? Are certain
	internal propellant structures more prone to failure than others? When do the most failures occur?
    
    \item \underline{Type} \\
    Reloadable $\geq$ single-use $\geq$ hybrid \\
	Considering a measure of simplicity, hybrid motors take the lowest rank and having reload capability - provided it
	doesn't add any significant risks - is better than not having reload capability.
    
    \item \underline{Uncertainty} \\
    Minimise \\
	Because we don't have the ability to throttle the motor, our final apogee is coupled to the motors performance: once
	it's been ignited, we have no active control. Therefore I think consistency of motor performance should have the
	highest priority of all parameters. We'd have room to configure the mass of the LV to raise or lower it's apogee
	accurately if we had accurate information about the motors performance. But mass optimisation wouldn't matter if
	there was still a significant chance the motor might leave the LV too high or too low. Adding mass to the vehicle
	might even amplify uncertainty in the apogee. How a value for 'consistency' is determined for each motor requires
	discussion. Which parameters should be used in this calculation? I think burn time, average thrust and the
	\textit{shape} of the thrust curve are priority (in that order). These would also be the most complex/expensive to
	test ourselves.
    
    % ========================================================================================================================
    
    \item \underline{Delay time(s)/whether it even has a delay charge or not (booster)} \\
    No delay charge $>$ delay charge, minimise delay time \\
	A miscalculation here would still give us a good chance of hitting the target apogee if everything else was correct,
	hence the low priority. This becomes relevant after motor burnout as the LV approaches its' apogee. As soon as the
	delay charge fires and the vehicle splits, it becomes impossible to predict its aerodynamic behaviour. If this fired
	too early it would make it less likely that we could accurately hit the target altitude. Therefore ideally,
	chute-deploy should be triggered by our own software and a separate charge. We could combine IMU measurements with a
	moving average of altitude to guarantee that the LV had passed apogee before activating the 'initial recovery event'
	(explained by docs). If all motors have delay charges built-in, we should choose the smallest value that gives us
	high confidence it will \textbf{not} activate before apogee. It would be possible to determine an optimum delay time
	if we had to choose one, by comparing the number of points gained for hitting the target altitude vs performing the
	initial recovery event successfully (without leaving it too late during descent and
	\href{https://www.rocketreviews.com/zipper--zippering-180703140820.html}{zippering} the LV), but this shouldn't be
	necessary.
    
	\todo[inline]{complete these}
    \item \underline{Average thrust} \\
		\todo[inline]{todo}
	Low average thrust, over a long burn time as opposed to high average thrust over short burn.

    \item \underline{Maximum thrust}
    
    \item \underline{Thrust-vs-time curve}
		\todo[inline]{look into stability tradeoff}
	Rule of thumb is at least 30mph clearing the launch mount,
    
    \item \underline{Total impulse and impulse class}
	\todo[inline]{todo}
    
    \item \underline{Burn time} \\
    Maximise \\
	Long burn time + low average thrust $\rightarrow$ minimised average veloctiy $\rightarrow$ minimised drag.
    
    \item \underline{Diameter} \\
    Minimise
    
    \item \underline{Length} \\
    Minimise
    
    \item \underline{Total mass} \\
    Minimise \\
	Although overall, the LV should be relatively heavy for momentum and resistance to wind disturbance, we want to have
	maximum control over the placement of mass (get everything essential done with the minimum mass and then inert mass
	to bring the total to our target. Also, for stability, the CoG of the LV should be closer to the nose than the fins
	and motor, so the mass of the motor should be minimised.
    
    \item \underline{Propellant mass (and how this changes over the burn)}
    
    \item \underline{Centre of gravity (and how this changes over the burn)}
    
    % ========================================================================================================================
    
    \item \underline{Price} \\
    Minimise
\end{enumerate}

\end{document}
